\documentclass[norsk, 10pt]{article}
\usepackage{babel}          % Ordelingsregler, osv
\usepackage[utf8]{inputenc}
\usepackage[T1]{fontenc}
\usepackage{booktabs}       % Ordentlige tabeller
\usepackage{url}            % Skrive url-er
\usepackage{textcomp}       % Den greske bokstaven micro i text-mode
\usepackage{units}          % Skrive enheter riktig
\usepackage{float}          % Figurer dukker opp der du ber om
\usepackage{lipsum}         % Blindtekst
\usepackage{amsmath, amsfonts, amssymb, amsthm}
\usepackage{caption,subfigure,listings, booktabs}
\usepackage{tikz,graphicx}
\usepackage{sectsty}

% Setter fonter
\usepackage{bbold,gillius}
\allsectionsfont{\sffamily} % Sans serif på alle overskrifter
%\renewcommand{\abstractname}{Executive Summary}
\captionsetup{width=.5\textwidth, textfont={small,it},labelfont={small,sf}}
\usepackage[sc,osf]{mathpazo} % Palatino


% Kodelisting
\usepackage{verbatim}
\lstset{language=matlab,breaklines=true,numbers=left} % For hele programmer.
%\lstinputlisting[language=matlab]{fil.m}

% Layout
%\usepackage[top=1.2in, bottom=1.7in, left=1.7in, right=1.7in]{geometry}
\usepackage[top=1.2in, bottom=1.7in, left=.7in, right=.7in]{geometry}
\frenchspacing % Rett mellomrom etter punktum.
\linespread{1.1} % Linjeavstand.
\usepackage[colorlinks=true]{hyperref} % Farge på lenker.

% Egendefinerte kommandoer
\newcommand{\dt}{\, {\rm d}t\, }
\newcommand{\dx}{\, {\rm d}x\, }
\newcommand{\dv}{\, {\rm d}v\, }
\newcommand{\dr}{\, {\rm d}r\, }
\newcommand{\dd}{\, {\text d} }
%\newcommand{\dp}{\ {\rm d}p\ }
\newcommand{\R}{\mathbb{R}}
\def\mean#1{\left\langle #1 \right\rangle}
\renewcommand{\exp}{\mathit{e}}
%\DeclareMathOperator{\dt}{dt}
\newcommand{\mb}[1]{\mathbf{#1}}
\def\para#1{\left( #1 \right)}
\newcommand{\ket}[1]{\left|#1\right\rangle}
\newcommand{\bra}[1]{\left\langle#1\right|}

%, trim = 1cm 7cm 1cm 7cm % PDF-filer som bilde

\begin{document}

% Forside
\begin{titlepage}
\begin{center}

\textsf{\Large FYS4150 - Computational Physics\\[0.5cm]
\rule{\linewidth}{0.5mm} \\[0.4cm]
{ \huge \bfseries  PROSJEKT 4}\\[0.10cm]
\rule{\linewidth}{0.5mm} \\[1.5cm]
{\Large Isingmodellen, Monte Carlo og parallellisering}}\\[1.5cm]
\textsc{}\\[1.5cm]

% Av hvem?

\textsf{\begin{minipage}{0.49\textwidth}
    \begin{center} \large
        Kandidat 72
    \end{center}
\end{minipage}}
%\begin{minipage}{0.49\textwidth}
%    \begin{center} \large
%        Anne-Marthe Hovda\\ \url{annemmho@uio.no} \\[0.8cm]
%    \end{center}
%\end{minipage}}


\vfill

% Dato nederst
\textsf{\large{Dato: \today}}

\end{center}
\end{titlepage}

\abstract{}
\\ \\


\section*{Introduksjon}

\section*{Teori}

\section*{Metode}

\section*{Resultat}

\begin{table}[H]
  \centering
  \begin{tabular}{ l l l l l}
    \toprule
    $I_{\text{Legendre}}$ & $I_{\text{Laguerre}}$ & $\epsilon_{\text{Legendre}}$ & $\epsilon_{\text{Laguerre}}$ & n \\
    \midrule
	0.129834 & 0.181567 & 0.326466 & 0.058093 & 10 \\
	0.199475 & 0.195887 & 0.034804 & 0.016192 & 15 \\
	0.177065 & 0.195636 & 0.081449 & 0.014892 & 20 \\
	0.189110 & 0.195240 & 0.018967 & 0.012837 & 25 \\
	0.185796 & 0.195070 & 0.036158 & 0.011955 & 30 \\
	\bottomrule
  \end{tabular}
  \caption{Resultat fra kjøringer av begge GK-metodene. Ingen av de er spesielt gode sammenliknet med Monte Carlo-metodene. Når vi hadde ti integrasjonspunkter så betydde det $10^6$ utregninger siden det var en seksdobbel løkke, en lite effektiv måte å angripe problemet på.}
  \label{tab:gauleg}
\end{table}

\section*{Numerisk stabilitet og presisjon}

\subsection*{Konklusjon}

\end{document}