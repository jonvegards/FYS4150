\documentclass[norsk, 12pt]{article}

\usepackage[T1]{fontenc}    % Riktig fontencoding
\usepackage[utf8]{inputenc} % Riktig tegnsett
\usepackage{babel}          % Ordelingsregler, osv
\usepackage{graphicx}       % Inkludere bilder
\usepackage{booktabs}       % Ordentlige tabeller
\usepackage{url}            % Skrive url-er
\usepackage{textcomp}       % Den greske bokstaven micro i text-mode
\usepackage{units}          % Skrive enheter riktig
\usepackage{float}          % Figurer dukker opp der du ber om
\usepackage{lipsum}         % Blindtekst
\usepackage{amsmath}        % Mattestæsj
\usepackage{listings}       % Kodetekst
\usepackage[a4paper,margin=1.7in]{geometry}

% JF i margen
\makeatletter
\renewcommand{\subsubsection}{\@startsection{subsubsection}{3}{-2cm}%
{-\baselineskip}{0.5\baselineskip}{\bf\large}}
\makeatother
\newcommand{\jf}[1]{\subsubsection*{JF #1}\vspace*{-2\baselineskip}}

% Skru av seksjonsnummerering
\setcounter{secnumdepth}{-1}

%, trim = 1cm 7cm 1cm 7cm % PDF-filer som bilde

\begin{document}

% Forside
\begin{titlepage}
\begin{center}

\textsc{\Large FYS4150 - Computational Physics}\\[0.5cm]
\rule{\linewidth}{0.5mm} \\[0.4cm]
{ \huge \bfseries  PROJECT 1}\\[0.10cm]
\rule{\linewidth}{0.5mm} \\[1.5cm]
\textsc{\Large svc}\\[1.5cm]
\textsc{}\\[1.5cm]

% Av hvem?
\begin{minipage}{0.49\textwidth}
    \begin{center} \large
        Henrik Sverre Limseth\\ \url{henrisli@uio.no} \\[0.8cm]
    \end{center}
\end{minipage}
\bigskip \\
\begin{minipage}{0.49\textwidth}
    \begin{center} \large
        Jon Vegard Sparre\\ \url{jonvsp@uio.no} \\[0.8cm]
    \end{center}
\end{minipage}
\begin{minipage}{0.49\textwidth}
    \begin{center} \large
        Anne-Marthe Hovda\\ \url{annemmho@uio.no} \\[0.8cm]
    \end{center}
\end{minipage}

\vfill

% Dato nederst
\large{Dato: \today}

\end{center}
\end{titlepage}

\abstract{I denne laboppgaven ble vi bedre kjent med resonans i en LC-krets og AM-radio. Vi så på dens frekvensrespons, fant resonansfrekvensene og simulerte til slutt et båndstoppfilter.}

\section*{Exercise a)}
\section*{Exercise b)}
\section*{Exercise c)}

% Draft for table over maximum error for different values of n
% n = 10: 		e = -5.95274
% n = 100: 		e = -2.68747
% n = 1000: 	e = -3.91861
% n = 10000: 	e = -4.95071
% n = 100000: 	e = -5.95274
% For n = 1e6 we get segmentation fault. And why does the error increase with n? #lol
\section*{Exercise d)}
\end{document}