\documentclass{tufte-book}
\usepackage{verbatim}
\usepackage[utf8]{inputenc}
\usepackage[T1]{fontenc}
\usepackage{amsmath, amsfonts, amssymb, amsthm}
\usepackage{textcomp}
\linespread{1.1} % Linjeavstand.

\newenvironment{loggentry}[2]% date, heading
{\noindent\textbf{#2}\marginnote{#1}\\}{\vspace{0.5cm}}
\title{Journal for FYS4150}
\author{Jon Vegard Sparre}

\begin{document}

\maketitle

\begin{loggentry}{03.09.2015}{Prosjekt 1}
Har produsert ein god del av \verb main.cpp, men har ikkje kome i hamn med LU-dekomponeringa.
\end{loggentry}

\begin{loggentry}{18.09.2015}{Prosjekt 2}
Har starta på prosjekt 2. Skal løyse Schrödingerlikninga for eit to-elektronsystem. Brukte dagens datalab på å lese gjennom oppgåva og finne ut kva som skulle gjerast og forstå Jacobis metode. Ved å diskretisere Schrödingerlikninga får vi ei eigenverdilikning $Ax = \lambda x$, som skal løysast med denne metoden. Metoden går ut på å diagonalisere matrisa $A$ ved å gange med $S^T$ frå venstre og $S$ frå høgre, $S$ er ei rotasjonsmatrise som blir bestemt av korleis $A$ ser ut. Vi vel så største ikkje-diagonale element i den nye $A$-matrisa til å vere null og finn rotasjonsvinkelen $\theta$ som så gir oss den nye matrisa $S$ (trur eg). Denne prosessen gjentakast $N$ gonger til vi har fått den diagonale matrisa. Når vi har den diagonale matrisa så har vi alle eigenverdiane på diagonalen, og det er dei vi er interessert i. Men samstundes som vi roterar $A$ så blir eigenvektorane endra, det må vi ta hensyn til på eit eller anna vis.

Eg starta på \verb|main.cpp| i går og fekk implementert mesteparten av algoritmen, men den verkar ikkje som den skal enda. Eg sat igjen med ei nesten diagonal matrise, den hadde veldig høge verdiar oppe i venstre hjørne, mens nede i høgre hjørne var verdiane i storleiksorden $10^{-15}$, som jo er ``null''.
\end{loggentry}

\end{document}